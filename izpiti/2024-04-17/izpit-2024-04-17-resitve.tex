\documentclass[arhiv]{../izpit}
\usepackage{fouriernc}
\usepackage{xcolor}
\usepackage{minted}


\begin{document}

\izpit{Programiranje II: poskusni izpit}{17.\ april 2024}{
  Čas reševanja je 60 minut.
  Veliko uspeha!
}

%%%%%%%%%%%%%%%%%%%%%%%%%%%%%%%%%%%%%%%%%%%%%%%%%%%%%%%%%%%%%%%%%%%%%%%

\naloga[\tocke{10}]

Za vsakega izmed spodnjih programov prikažite vse spremembe sklada in kopice, če poženemo funkcijo \mintinline{rust}{main}. Za vsako spremembo označite, po kateri vrstici v kodi se zgodi.

\podnaloga
\begin{minted}[linenos]{rust}
  fn f(a: i32, b: i32) -> i32 {
    a * b
  }
  fn g(x: i32) -> i32 {
    f(x, x + 1)
  }
  fn main() {
    let m = 6;
    let n = g(m);
    println!("{n}")
  }
\end{minted}

\podnaloga
\begin{minted}[linenos]{rust}
  fn f(s: String) {
    println!("{s}")
  }
  fn g(s: String) {
    f(s)
  }
  fn main() {
    let s2 = String::from("2");
    let s1 = String::from("4");
    if true {
      println!("{s2}");
    }
    g(s1);
  }
\end{minted}

\podnaloga
\begin{minted}[linenos]{rust}
  fn f(s: &String) {
    println!("{s}")
  }
  fn g(s: String) {
    f(&s)
  }
  fn main() {
    let s1 = String::from("4");
    let s2 = String::from("2");
    g(s1);
    println!("{s2}");
  }
\end{minted}


%%%%%%%%%%%%%%%%%%%%%%%%%%%%%%%%%%%%%%%%%%%%%%%%%%%%%%%%%%%%%%%%%%%%%%%

\naloga[\tocke{10}]

Definirajmo tip množic \mintinline{rust}{Set<T>}. Dopolnite signature spodnjih metod. Če v dani prostor ni treba dopisati ničesar, ga prečrtajte.

\podnaloga
\mintinline{rust}{fn contains(_____ self, x: ______) ________}, ki preveri, ali dana množica vsebuje element \mintinline{rust}{x}.

\noindent
\emph{Rešitev:} \mintinline{rust}{fn contains(&self, x: &T) -> bool}.

\podnaloga
\mintinline{rust}{fn power_set(_____ self) ________}, ki vrne potenčno množico dane množice.

\noindent
\emph{Rešitev:} \mintinline{rust}{fn power_set(&self) -> Set<Set<T>>}. 

\podnaloga
\mintinline{rust}{fn intersection(_____ self, other: ______) _______}, ki izračuna presek dveh množic.

\noindent
\emph{Rešitev:} \mintinline{rust}{fn intersection(&self, other: &Self) -> Self} (namesto \mintinline{rust}{Self} lahko pišete tudi \mintinline{rust}{Set<T>}).

\podnaloga
\mintinline{rust}{fn add(_____ self, x: ______) ________}, ki v obstoječo množico doda element \mintinline{rust}{x}.

\noindent
\emph{Rešitev:} \mintinline{rust}{fn add(&mut self, x: T)}.

\podnaloga
\mintinline{rust}{fn into_iter(_____ self) ________}, ki iz množice naredi iterator po njenih elementih.

\noindent
\emph{Rešitev:} \mintinline{rust}{fn into_iter(self) -> Iter<T>}.


%%%%%%%%%%%%%%%%%%%%%%%%%%%%%%%%%%%%%%%%%%%%%%%%%%%%%%%%%%%%%%%%%%%%%%%

\naloga[\tocke{30}]

Za vsakega izmed spodnjih programov:
\begin{enumerate}
  \item razložite, zakaj in s kakšnim namenom Rust program zavrne;
  \item program popravite tako, da bo veljaven in bo učinkovito dosegel prvotni namen.
\end{enumerate}

\podnaloga
\begin{minted}{rust}
fn main() {
    let v = vec![1, 2, 3];
    for x in v {
      v.push(x);
    }
}
\end{minted}

\noindent
\emph{Rešitev:} Vektor \mintinline{rust}{v} je izposojen za zanko \texttt{for}, zato si ga ne moremo izposoditi še za spreminjanje za metodo \mintinline{rust}{push}, saj bi spreminjanje lahko \mintinline{rust}{v} prestavilo na drugo mesto v kopici, zanka \texttt{for} pa bi ostala s kazalcem na neveljaven del pomnilnika. Če želimo vektor s tako zanko podvojiti, ga moramo prekopirati z:
%
\begin{minted}{rust}
  fn main() {
      let v = vec![1, 2, 3];
      for x in v.clone() {
        v.push(x);
      }
  }
\end{minted}

\podnaloga
\begin{minted}{rust}
enum Drevo {
    Prazno,
    Sestavljeno(Drevo, u32, Drevo),
}
\end{minted}

\noindent
\emph{Rešitev:} Privzeto Rust variante v pomnilniku predstavi z oznako izbrane variante, ki ji zaporedoma sledijo predstavitve posameznih argumentov. Ob taki predstavitvi je varianta \mintinline{rust}{Sestavljeno} lahko poljubno velika, zato zanjo ni mogoče rezervirati pomnilnika. Rešitev je v tem, da obe poddrevesi shranimo na kopico, zato je varianta \mintinline{rust}{Sestavljeno} predstavljena samo z dvema kazalcema in vrednostjo v korenu.
%
\begin{minted}{rust}
  enum Drevo {
    Prazno,
    Sestavljeno(Box<Drevo>, u32, Box<Drevo>),
}
\end{minted}

\podnaloga
\begin{minted}{rust}
fn daljsi_niz(s1: &str, s2: &str) -> &str {
    if s1.len() > s2.len() {
        return s1;
    } else {
        return s2;
    }
}
\end{minted}

\noindent
\emph{Rešitev:} Rust ne more določiti življenjske dobe vrnjene reference, ki bi zagotavljala dostop do rezultata ne glede na to, katerega od argumentov se izbere. Zato moramo dobe eksplicitno napisati:
%
\begin{minted}{rust}
  fn daljsi_niz(s1: &'a str, s2: &'a str) -> &'a str {
    ...
  }
\end{minted}

\podnaloga

\begin{minted}{rust}
fn g(s1: &String, s2: &String) -> () {
    /// Poljubna koda, da je tip funkcije ustrezen
}

fn main() {
    let mut s1 = String::from("1");
    g(&mut s1, &s1);
}
\end{minted}

\noindent
\emph{Rešitev:} Tako kot pri prvi podnalogi imamo hkrati spremenljivo in nespremenljivo referenco na niz \mintinline{rust}{s1}. Glede na to, da funkcija \mintinline{rust}{g} ne potrebuje spremenljivega dostopa, lahko preprosto pišemo
\begin{minted}{rust}
  fn main() {
      let mut s1 = String::from("1");
      g(&s1, &s1);
  }
  \end{minted}

\podnaloga

\begin{minted}{rust}
fn vsebuje<T>(v: &Vec<T>, x : &T) -> bool {
    for y in v {
      if x == y {
        return true
      }
    }
    return false
}
\end{minted}

\noindent
\emph{Rešitev:} Ker primerjamo \mintinline{rust}{x} in \mintinline{rust}{y}, mora tip \mintinline{rust}{T} podpirati vsaj značilnost \mintinline{rust}{PartialEq}, saj sicer Rust nima na voljo kode, ki naj bi se izvedla ob primerjavi. Zahtevo po značilnosti moramo zapisati v tip:
\begin{minted}{rust}
  fn vsebuje<T : PartialEq>(v: &Vec<T>, x : &T) -> bool {
    ...
  }
\end{minted}

\end{document}
